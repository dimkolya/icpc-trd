\section{Math}

\subsection{NT}
\subsubsection{Extended euclidean}
$$ax+by=\text{gcd}(a,b),\quad x=x_0+n\dfrac{b}{g},\;y=y_0-n\dfrac{a}{g},\;n\in\mathbb{Z}.$$
\inputminted{cpp}{../code/math/egcd.cpp}
\subsubsection{Binpow}
\inputminted{cpp}{../code/math/binpow.cpp}
\subsubsection{Euler function $\varphi$}
\inputminted{cpp}{../code/math/euler.cpp}
\subsubsection{$\varphi$ function for 1 to n}
\inputminted{cpp}{../code/math/euler_precalc.cpp}
\subsubsection{Miller-Rabin test}
\inputminted{cpp}{../code/math/miller_rabin.cpp}

\subsection{Polynomial multiplication}
$c_k=\displaystyle\sum_{i=0}^ka_ib_{k-i}\le(k+1)\cdot\text{MAX}^2$

\subsubsection{FFT}
Complex or modular $\{w^i\}_{i=0}^{n-1},\;w^n=1$.
\begin{enumerate}
\item $\forall i:\;A(w^i),B(w^i)$
\item $\forall i:\;y_i=A(w^i)B(w^i)$
\item $c_i=Y(w^{-i})\cdot n^{-1}$
\end{enumerate}
\inputminted{cpp}{../code/math/fft.cpp}
\subsubsection{Karatsuba}
$O(N^{\log_2 3})\approx O(N^{1.58\dots})$
\inputminted{cpp}{../code/math/karatsuba.cpp}

\subsection{Формулы}
\subsubsection{Ряды}
$$1+2+\dots+n=\frac{n(n+1)}{2},$$
$$1^2+2^2+\dots+n^2=\frac{n(n+1)(2n+1)}{6},$$
$$1^3+2^3+\dots+n^3=\frac{n^2(n+1)^2}{4},$$
$$1^4+2^4+\dots+n^4=\frac{n(n+1)(2n+1)(3n^2+3n-1)}{30}.$$
\textit{Треугольные} $S_n=1+3+6+\dots+\frac{n(n+1)}{2}=\frac{n(n+1)(n+2)}{6}.$
\subsubsection{Сочетания}
\textit{Треугольник Паскаля.} $C_n^k=C_{n-1}^k+C_{n-1}^{k-1}$.\\
\textit{Вандермонда.} $C_{n_1+\dots+n_p}^m=\displaystyle\sum_{k_1+\dots+k_p=m}C_{n_1}^{k_1}\cdot\ldots\cdot C_{n_p}^{k_p}$.\\
\textit{Hockey-stick identity.} $C_r^r+C_{r+1}^r+\dots+C_n^r=C_{n+1}^{r+1}$.\\
\textit{Числа Фибоначчи.} $F_n=\sum C_{n-k}^k$.\\
\subsubsection{Асимптотика}
\textit{Стирлинга.} $\ln n!=n\ln n-n+O(\ln n).$
$$n!=\sqrt{2\pi n}\left(\dfrac{n}{e}\right)^n\exp\dfrac{1}{12n+\theta_n},\;\theta_n\in(0,1).$$\\
\subsubsection{Код Грея}
$n\otimes (n>>1)$\\

\subsection{Дискретная математика}
\subsubsection{Числа Стирлинга}
1. I-го рода, количество перестановок порядка $n$ с $k$ циклами: $S[0,0]=1,S[0,k]=S[n,0]=0,S[n,k]=S[n-1,k-1]+(n-1)S[n-1,k]$;\\
2. II-го рода, количество разбиений множества из $n$ элементов на $k$ непустых подмножеств: $S[n,k]=kS[n-1,k]+S[n-1,k-1]$.\\
\subsubsection{Разбиение на слагаемые}
1. Порядок важен: $2^{n-1}$;\\
2. Порядок не важен: $p[n,k]$ -- кол-во слагаемых, где максимум $\le k$, $p[n,k]=p[n-k,k]+p[n,k-1]$.\\
\begin{minted}{cpp}
if(n<0)return 0;if(n==0)return 1;if(k==0)return 0;
\end{minted}
\subsubsection{Формула Эйлера}
$V-E+F=2$.\\
\subsubsection{Два ферзя}
$\frac{n^4}{2}-\frac{5n^3}{3}+\frac{3n^2}{2}-\frac{n}{3}$.\\
\subsubsection{Лемма Бернсайда}
$C=\dfrac{1}{|G|}\displaystyle\sum_{g\in G}l^{}P(g)$, где $C$ --- кол-во различных классов эквивалентности, $P(g)$ --- кол-во циклов в перестановке $g$, l --- кол-во различных состояний одного элемента.

\subsection{Линейная алгебра}
\subsubsection{Обратные матрицы}
\(\left(
\begin{array}{cc}
    x_1 & y_1 \\
    x_2 & y_2
\end{array}\right)^{-1}=\displaystyle\frac{1}{det}\left(
\begin{array}{cc}
    y_2 & -y_1 \\
    -x_2 & x_1
\end{array}\right),\ x_1y_2-x_2y_1;\)
\(\displaystyle\frac{1}{det}\left(
\begin{array}{ccc}
    y_2z_3-y_3z_2 & y_3z_1-y_1z_3 & y_1z_2-y_2z_1 \\
    x_3z_2-x_2z_3 & x_1z_3-x_3z_1 & x_2z_1-x_1z_2 \\
    x_2y_3-x_3y_2 & x_3y_1-x_1y_3 & x_1y_2-x_2y_1
\end{array}\right),\)
\(x_1y_2z_3+y_1z_2x_3+z_1x_2y_3-z_1y_2x_3-y_1x_2z_3-x_1z_2y_3;\)
\subsubsection{Числа Фибоначчи}
\[f_{2n}=f_n(2f_{n+1}-f_n),\;f_{2n+1}=f_{n+1}^2+f_n^2\]
\[\left(\begin{array}{cc}
    1 & 1 \\
    1 & 0
\end{array}\right)^n=\left(\begin{array}{cc}
    f_{n + 1} & f_n \\
    f_n & f_{n - 1}
\end{array}\right)\]

\subsection{Геометрия}

\subsubsection{Pick's formula for area}
\(S=I+\displaystyle\frac{B}{2}-1\);\\
\subsubsection{Cross product}
\(\vec a\times\vec b=-\vec b\times\vec a,\ |\vec a\times\vec b|=|\vec a||\vec b|\sin\varphi\),
\[\left\{\begin{array}{l}
    c_x=a_yb_z-a_zb_y \\
    c_y=a_zb_x-a_xb_z \\
    c_z=a_xb_y-a_yb_x
\end{array}\right.,\quad\vec c=\left|\begin{array}{ccc}
    \vec i & \vec j & \vec k \\
    a_x & a_y & a_z \\
    b_x & b_y & b_z
\end{array}\right|;\]
\subsubsection{Volume by 4 points}
\(V=\displaystyle\frac{|(\vec a-\vec d)\cdot((\vec b-\vec d)\times(\vec c-\vec d))|}{6}\);\\
\subsubsection{Angle between vectors}
\(\varphi=\) atan2\((\vec a\times\vec b,\vec a\cdot\vec b)\);\\
\subsubsection{Plane in space}
1. \(Ax+By+Cz+D=0,\ \vec n = (A,B,C)\);\\
2. By point and \(\vec n\): \(A(x-x_0)+B(y-y_0)+C(z-z_0)=0\);\\
\subsubsection{Which side of a plane a point \(\vec p\)}
\(\vec q = (\vec b - \vec a)\times(\vec c - \vec a)\). Result is a sign of \((\vec p - \vec a)\times\vec q\), equals 0 if \(p\in (a,b,c)\);
\subsubsection{Point in angle}
\((C\in\angle AOB)\): \(sign(\vec{OA}\times\vec{OC})=sign(\vec{OA}\times\vec{OB})\) \& \(sign(\vec{OB}\times\vec{OC})=sign(\vec{OB}\times\vec{OA})\);
\subsubsection{Rays intersection}
\((\vec{AB}\cap\vec{CD})\): \(sign(\vec{AB}\times\vec{AC})\neq sign(\vec{AB}\times\vec{CD})\) \& \(sign(\vec{CD}\times\vec{CB})\neq sign(\vec{CD}\times\vec{AB})\);
\subsubsection{Segments intersection}
\(sign(\vec{AB}\times\vec{AC})\cdot sign(\vec{AB}\times\vec{AD})\le0\) \& mirror condition for non-collinear;
\begin{minted}{cpp}
if (((a - b) ^ (c - d)) != 0) {
  return sign((a-b)^(a-c))*sign((a-b)^(a-d))<=0
      && sign((c-d)^(c-a))*sign((c-d)^(c-b))<=0;
}
if (((a-c)^(a-d))!=0 || ((c-a)^(c-b)) != 0) {
  return false;
}
return (c-a)*(c-b)<=0 || (d-a)*(d-b)<=0
       || (a-c)*(a-d)<=0 || (b-c)*(b-d)<=0;
\end{minted}
\subsubsection{Lines intersection...}
Let \(A_1x+B_1y+C_1=0\) and \(A_2x+B_2y+C_2=0\). They match if \(\displaystyle\frac{A_1}{A_2}=\frac{B_1}{B_2}=\frac{C_1}{C_2}\), parallel if \(\displaystyle\frac{A_1}{A_2}=\frac{B_1}{B_2}\neq\frac{C_1}{C_2}\), else \(x=\displaystyle\frac{C_2B_1-C_1B_2}{A_1B_2-A_2B_1},\ y=\frac{C_2A_1-C_1A_2}{A_2B_1-A_1B_2}\);
\subsubsection{Distance between point \(C\) and line \(AB\)}
1. \(h_{AC}=\displaystyle\frac{|\vec{AB}\times\vec{AC}|}{|\vec{AB}|}\);\\
2. \(d=\displaystyle\frac{|Ax_0+By_0+C|}{\sqrt{A^2+B^2}}\).\\
\subsubsection{Projection of a point onto a line}
\((x_0\pm d\displaystyle\frac{A}{\sqrt{A^2+B^2}},y_0\pm d\displaystyle\frac{B}{\sqrt{A^2+B^2}})\);\\
\subsubsection{Line coefficients by 2 points}
\(A=y_2-y_1,\ B=x_1-x_2,\ C=x_2y_1-x_1y_2\);\\

\subsection{Теория вероятности}
\subsubsection{Дискретные распределения}
\textit{Бернулли.} Вероятность успеха $p$:
$$\xi\in\{0,1\}\quad \mathbb{E}\xi=p,\;\mathbb{V}\xi=p(1-p).$$
\textit{Биномиальное.} Сумма независимых бернул. сл. величин:
$$\xi_k\sim\text{Bern}(p),\;\eta\sim\text{Bin}(n),\;\eta=\xi_1+\dots+\xi_n.$$
$$P(\eta=k)=C_n^kp^k(1-p)^{n-k},\;\mathbb{E}\eta=np,\;\mathbb{V}\eta=np(1-p).$$
\textit{Пуассона.} $k$ событий за $T$, $\lambda$ ср. кол-во событий за $T$:
$$P(\xi=k)=e^{-\lambda}\dfrac{\lambda^k}{k!},\;\mathbb{E}\xi=\mathbb{V}\xi=\lambda.$$
\textit{Теорема.} Пусть $\xi\sim\text{Bin}(n,\frac{\lambda}{n})$, тогда:
$$\displaystyle\lim_{n\to\infty}P(\xi=k)=e^{-\lambda}\dfrac{\lambda^k}{k!},\quad k\in\mathbb{Z}_+.$$
\textit{Геом.} До первого успеха (включительно) $k$ попыток:
$$P(\xi=k)=(1-p)^{k-1}p,\;\mathbb{E}=\dfrac{1}{p},\;\mathbb{V}\xi=\dfrac{1-p}{p^2}.$$
\textit{Гипергеом.} Среди $N$ деталей $K$ бракованных. Вероятность того, что среди $n$ выбранных ровно $k$ бракованных:
$$p_k=\dfrac{C_K^kC_{N-K}^{n-k}}{C_N^n},\;\mathbb{E}\xi=\dfrac{nK}{N},\;\mathbb{V}\xi=\dfrac{nK(N-K)(N-n)}{N^2(N-1)}.$$